\subsection{Communications and Radar Spectrum Sharing}
CRSS is the technique that allows the communications and radar systems 
to work in the same frequency band. It can be categorized into coexistence 
and co-design

CRSS is motivated by solving the coexistence problem of radar and 
communication systems \cite{feng2020survey}. In the coexistence cases, the most significant 
issue is the interference cancellation, so that the two systems can 
work without imposing interference on each other. Many methods have been investigated, such as 
optimizing the beamforming \cite{rihan2018optimum,qian2018joint,li2016optimum,li2017joint,liu2018mimo}, 
nulling interference at transmitter \cite{sodagari2012projection, mahal2017spectral}, 
and adaptively eliminating interference at the receiver \cite{deng2013interference,meager2016estimation,nartasilpa2018let}.

Co-design is to integrate the two systems into a single platform and design a dual-functional radar and communications (DFRC) system,
where target sensing and communication can be performed simultaneously \cite{Hassanien2019DFRC}. There are two methods to achieve it:
radar- and communication-centric.

The main idea of radar-centric DFRC is embedding the information into the radar waveform, without
interfering with the radar function. For example, the information can be embedded using beamforming modulation and index modulation \cite{Hassanien2019DFRC}. 
Hassanien et al. \cite{Hassanien2018} considered the two-way communication where the communication information is embedded in 
the radar signal emitting in the direction of users.
However, the rate of communications is restricted by the pulse repetition frequency (PRF) of radar,
because each radar pulse can only support a small number of communication symbols \cite{liu2020beamforming}.   

In communication-centric DFRC, the target data is extracted from the echoed communication waveform. 
All the commonly-used communication waveforms can be used for this task \cite{liu2020tutorial}. 
As opposed to the radar-centric method, the communication-centric method 
is capable of supporting a higher communication rate and guaranteeing radar performance by carefully designing the waveform \cite{liu2020beamforming}. 
Therefore, in this project, we mainly focus on waveform design for communication-centric DFRC.

Many works have been done in the field of communication-centric DFRC \cite{liu2018beamforming,liu2020beamforming,xu2020tradeoff}. 
For example, in \cite{liu2018beamforming}, Liu et al. designed a DFRC platform that supports probing several radar targets and serving several communication 
users simultaneously. Specifically, two platform deployments were proposed: separated deployment and shared deployment. In the former, 
the antennas are separated into two groups and each group performs radar or communication function, while shared 
deployment only transmits communication waveform which is designed to approximate a pre-designed beampattern or 3dB-main beamwidth. 
The shared deployment was also considered in \cite{liu2020beamforming}, but an additional jointly precoded
radar waveform is added on communication waveform to maximize the degree of freedom (DoF) of radar. Apart from achieving the desired beampattern, the waveform was also 
designed to minimize the mean-squared cross-correlation, which can suppress the interference from undesired targets. However, in 
both \cite{liu2018beamforming} and \cite{liu2020beamforming}, the SINR was used as the communication metric, instead of more representative 
metrics like weighted sum rate (WSR). Xu et al. \cite{xu2020tradeoff} first considered WSR maximization in DFRC context, where the 
tradeoff between probing power at target and WSR were investigated. Compared with beampattern approximation in \cite{liu2018beamforming} and \cite{liu2020beamforming},
probing power maximization lead to clearer tradeoff comparison. Thus,
the probing power at target and WSR were used as radar and communication metrics in this project.

% Liu et al. \cite{liu2018beamforming} designed a joint beamforming method, which 
% is capable of formulating the desired beampattern of radar function under the constraints of SINR and power of downlink communication.
% Apart from the desired beampattern, an SDR-based optimization algorithm proposed by Liu et al. \cite{liu2020beamforming} also 
% minimizes the cross correlation pattern of signal in different directions. Meanwhile, another sub-optimal zero forcing based method 
% is proposed, which has lower complexity. Xu et al. \cite{xu2020tradeoff} considered the most representative communication 
% metric Weighted Sum Rate (WSR) instead of the SINR. They also exploited the probing power to replace desired beampattern as the
% radar metric, which can reveal the trade-off relationship in DFRC system more clearly. A WMMSE and SDP based optimization algorithm 
% was proposed to maximize the WSR and probing power simultaneously.

\subsection{Reconfigurable Intelligent Surface}
As was aforementioned, many works considered the joint beamforming design for RIS-aided communication system \cite{wu2019IRS,guo2019WSR,guo2020ris}.
Specifically, a RIS-aided wireless communication was investigated in \cite{wu2019IRS}, where both single- and multiple-user cases are considered. 
With the additional RIS-aided channel between BS and users, the authors tried to use as low transmit power as possible but guaranteed the SINR at users. 
Numerical results indicate that the number of RF chains can be significantly reduced in a RIS-aided MIMO system 
when the performance is the same as massive MIMO. Similarly, WSR is also more representative than SINR in the context of RIS-aided communication system.
Therefore, Guo et al. \cite{guo2019WSR} first studied the maximization of WSR for all users in the context of RIS-aided communication system.
To solve the non-convex and NP-hard WSR maximization problem, an algorithm using fractional programming (FP) \cite{shen2018fractional,shen2018fractional2} and alternating optimization (AO) was proposed.
The work \cite{guo2020ris} extended the method of \cite{guo2019WSR}, where the beamformings are jointly designed to maximize the WSR with imperfect channel state information (CSI).

% As was aforementioned, many works considered the joint beamforming design at RIS-aided communication system \cite{wu2019IRS,guo2019WSR,guo2020ris}.
% Specifically, Wu and Zhang \cite{wu2019IRS} investigated RIS-aided wireless system in both single and multiple user cases, 
% where the active and passive beamforming are jointly designed to minimize transmit power under the constraint of SINR at users. 
% Numerical results show that the number of RF-chains can be significantly reduced in RIS-aided MIMO system 
% when the performance is the same as massive MIMO. Similarly, WSR is also more representative than SINR in context of RIS-aided communication system.
% Therefore, Guo et al. \cite{guo2019WSR} first studied the maximization of WSR for all users by jointly designing active and passive beamforming.
% Although WSR maximization problem is NP-hard, an algorithm was proposed to solve it based on fractional programming \cite{shen2018fractional,shen2018fractional2} and alternating optimization.
% The work \cite{guo2020ris} extended the method of \cite{guo2019WSR}, where the beamforming are joint designed to maximize the WSR when 
% the channel state information (CSI) is imperfect.

% More specifically, Wu and Zhang \cite{wu2019IRS} proposed an Alternating Optimization based methods to jointly optimize transmit beamforming at BS and 
% passive beamforming at RIS in both single and multiple user cases. Numerical results shows that transmit power at BS is significantly
% reduced in RIS-aided MIMO system when the same performance is achieved as massive MIMO system. 
% Guo et al. \cite{guo2019WSR} considered the WSR as the communication metric and proposed a two-stage algorithm based on Fractional Programming
% to jointly optimize transmit and passive beamforming. The imperfect CSI is investigated in \cite{guo2020ris}.

Recently, RIS also shows benefits in aiding radar system \cite{lu2021intelligent,buzzi2021radar,aubry2021reconfigurable,buzzi2021foundations}.
In \cite{lu2021intelligent}, the active beamforming for target detection at MIMO radar and passive beamforming at a co-located RIS are optimized
separately, in order to increase the detection performance. A RIS-aided detection algorithm was also proposed and it is shown that RIS can result in 
better Cramér–Rao bound (CRB). Unlike \cite{lu2021intelligent} wherein the RIS helps to receive the reflected signal at radar receiver, 
Buzzi et al. \cite{buzzi2021radar} designed a system wherein the radar transmits or receives through two beams, one towards the target and another towards RIS.
It is theoretically proved that RIS can provide improvement to received SNR by phase-aligning the echo signals no matter the radar and RIS are closely-
or widely-spaced. When the LOS between radar and target is blocked, an artificial path can be established via RIS. Aubry et al. \cite{aubry2021reconfigurable}
investigated this scenario by discussing the data acquisition, resolution issues of estimated parameters, and size and system parameters of RIS.
By considering all the aforementioned RIS-aided scenarios, a general signal model was introduced in \cite{buzzi2021foundations}, which can 
cover the following conditions: one or two RIS's, mono-static or bi-static radar configuration, and with or without LOS. They also 
optimized the passive beamforming at RIS to improve the detection probability with the fixed false-alarm probability.

In most existing investigations about RIS-aided systems, including the aforementioned ones, the single connected reconfigurable impedance network is widely applied to model 
the RIS. Recently, Shen et al. \cite{shen2020modeling} proposed a generalized model in which the RIS network is group or fully connected. 
This novel model significantly outperforms the single connected RIS network in terms of received power. This project also investigated the potential benefits of this generalized
RIS model in the RIS-aided DFRC system.

\subsection{RIS-aided DFRC}
The RIS-aided DFRC system is still an open research area with few works \cite{wang2020ris, jiang2021dfrc,wang2021joint}.
Wang et al. \cite{wang2020ris} first investigate a RIS-aided radar-communication coexistence, where the radar detection probability
is maximized by optimizing active and passive beamforming and the received SINR at communication users is guaranteed.
It was proved that the detection probability maximization problem can be relaxed to a interference minimization problem of received echos at radar. 
In \cite{jiang2021dfrc}, the DFRC system is first considered. Similarly, the SINR of echos is maximized under users' SINR constraints.
A two-stage algorithm was designed for the non-convex joint beamforming optimization and the improvement of 
SINR in both radar and communication. RIS-aided DFRC system was also studied in \cite{wang2021joint}.
Unlike the previous two works, the beamforming is designed to form a pre-designed beampattern and eliminate the multi-user interference (MUI).
The tradeoff between radar and communication performance was also investigated. Nevertheless, as mentioned before, WSR is a more representative metric than SINR and MUI
in the context of both DFRC and RIS-aided system. Thus, in this project, the objective is to maximize the WSR at users in RIS-adied DFRC system.

% The RIS-aided DFRC system is still an open research area with few works \cite{wang2020ris, jiang2021dfrc,wang2021joint}.
% Wang et al. \cite{wang2020ris} first investigate a RIS-aided radar-communication coexistence system and the active and passive
% beamforming is jointly designed to maximize radar detection probability under users' SINR constraints. It was proved that the detection 
% probability maximization problem can be relaxed to a interference minimization problem of received echos at radar. In \cite{jiang2021dfrc}, the 
% DFRC system is first considered. Similarly, the SINR of echos is maximized under users' SINR constraints.
% A two-stage algorithm was proposed to solve the non-convex active and passive beamforming optimization problem and the improvement of 
% SINR in both radar and communication systems was shown by the simulation results. RIS-aided DFRC system was also studied in \cite{wang2021joint}.
% Unlike the previous two works, the beamforming is designed to achieve a desired beampattern and minimize the multi-user interference (MUI).
% The tradeoff between radar and communication performance was also investigated. Nevertheless, as mentioned before, WSR is a more representative metric than SINR and MUI
% in context of both DFRC and RIS-aided system. Thus, in this project, the objective is to maximize the WSR at users in RIS-adied DFRC system.