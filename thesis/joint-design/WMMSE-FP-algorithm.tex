It is clear that both \eqref{problem:seperated_problem} and \eqref{problem:shared_problem} are non-convex optimization problem.
However, we point out that they can be converted to
two optimization problems with respect to $\{{\bf{P}},{\bf{R}}_x\}$ and
$\bf{\Theta}$, respectively, using Weighted Minimum Mean Square Error (WMMSE) 
framework \cite{christensen2008weighted} and Fractional 
Programming (FP) \cite{shen2018fractional}. Then the optimal solution can be obtained in an alternative manner.

\subsection{Algorithm for Separated Deployment}

\subsubsection{WMMSE for Active Beamforming} \label{sec:WMMSE_separated}

Based on the WMMSE method proposed in \cite{christensen2008weighted}, the WSR 
maximization problem in \eqref{problem:seperated_problem} can be converted to an equivalent weight mean square
error (MSE) minimization problem via the following process.

The observation at user $k$ can be rewritten as
\begin{align} \label{eqn:separated_received_signal_transformed}
    x_k  & = {\bf c}_k^H \sum_{j=1}^K {\bf p}_j s_j +  {\bf r}_k^H {\bf q} + n_k \nonumber
    \\ & = {\bf c}_k^H {\bf p}_k s_k  + \underbrace{{\bf c}_k^H \sum_{j=1, j \neq k}^K {\bf p}_j s_j +  {\bf r}_k^H {\bf q}}_{\text{interference}} + n_k .
\end{align}
So the efficient noise plus interference power is 
\begin{equation} \label{eqn:separated_noise_interference}
    N_k=\sum_{j=1, j\neq k}^K |{\bf c}_k^H {\bf p}_j|^2 + {\bf r}_k^H {\bf R}_{\bf q} {\bf r}_k + \sigma^2_n.
\end{equation}
The estimated symbol using the receiver $g_k$ at user $k$ is
\begin{align} \label{eqn:separated_estimated}
    \hat{s}_k &= g_k x_k \nonumber\\
    &= g_k {\bf c}_k^H \sum_{j=1}^K {\bf p}_j s_j + g_k {\bf r}_k^H {\bf q} + g_k n_k.
\end{align}
The MSE is given as
\begin{align} \label{eqn:separated_mse}
    e_k  = & \mathbb{E} \Big[ \Vert \hat{s}_k - s_k \Vert^2 \Big] \nonumber\\
    = & |g_k|^2 \left( \sum_{j=1}^K | {\bf c}^H_k {\bf p}_j |^2 + {\bf r}_k^H {\bf R}_{\bf q} {\bf r}_k + \sigma_n^2 \right) - 2 \mathrm{Re}\left\{ g_k {\bf c}_k^H {\bf p}_k \right\} + 1.
\end{align}

The solution to the following unconstrained convex optimization problem yields the MMSE receiver.
\begin{align} \label{eqn:separated_mmse_receiver}
    g_k^{\mathrm{MMSE}} & = \arg \min_{g_k} e_k \nonumber
    \\ & = \frac{{\bf p}_k^H {\bf c}_k}{ | {\bf c}_k^H {\bf p}_k|^2 + N_k} \nonumber
    \\ & = \frac{{\bf p}_k^H {\bf c}_k}{\sum_{j=1, }^K |{\bf c}_k^H {\bf p}_j|^2 + {\bf r}_k^H {\bf R}_{\bf q} {\bf r}_k + \sigma^2_n}.
\end{align}
where $\partial e_k / \partial g_k = 0$ is the condition of the 
optimal MMSE receiver.

The MSE at the output of MMSE receiver is 
\begin{align} \label{eqn:separated_mmse_error}
    e_k^{\mathrm{MMSE}} & = \mathbb{E} \Big[(g_k^{\mathrm{MMSE}}x_k - s_k )(g_k^{\mathrm{MMSE}}x_k - s_k )^* \Big] \nonumber
    \\ & = \left(1+\frac{| {\bf c}_k^H {\bf p}_k|^2}{N_k} \right)^{-1} \nonumber
    \\ &= 1 - \frac{| {\bf c}_k^H {\bf p}_k|^2}{\sum_{j=1, }^K |{\bf c}_k^H {\bf p}_j|^2 + {\bf r}_k^H {\bf R}_{\bf q} {\bf r}_k + \sigma^2_n} .
\end{align} 

We can formulate another MSE minimization and probing power maximization problem, which is
\begin{subequations} \label{problem:separated_mse_minimization}
    \begin{align}
        \min_{\mathbf{P},\mathbf{R}_{\bf q}} \quad &  \rho \sum_{k=1}^K w_k e_k(\mathbf{P},\mathbf{R}_{\bf q}) -\mathbf{a}_r^H(\varphi_m)\mathbf{R}_{\bf q} \mathbf{a}_r(\varphi_m) - \mathbf{a}_c^H(\varphi_m)\mathbf{P}\mathbf{P}^H\mathbf{a}_c(\varphi_m)\\
        \mathrm{s.t.} \quad &\mathrm{diag}(\mathbf{R}_{\bf q}) = \frac{P_r \mathbf{1}^{M_r \times 1}}{M_r}, \\ 
        &\mathrm{Tr} (\mathbf{P}\mathbf{P}^H) \leq P_c, \\ 
        & \mathbf{R}_{\bf q} \succeq 0, {\bf R}_{\bf q} = {\bf R}_{\bf q}^H ,
    \end{align}
\end{subequations}
According to \cite{christensen2008weighted}, the problem \eqref{problem:separated_mse_minimization} has the 
same optimal solution of $\{{\bf{P}},{\bf{R}}_{\bf q}\}$ as \eqref{problem:seperated_problem} for fixed ${\bf \Theta}$ as long as
\begin{equation} \label{eqn:seperated_wmmse_weight}
    w_k = \mu_k (e_k^{\mathrm{MMSE}})^{-1}.
\end{equation}

Problem \eqref{problem:separated_mse_minimization} is still non-convex because of the last term. Based on the
transformation method in \cite{xu2020tradeoff}, the last term is equal to
\begin{align} \label{eqn:separated_transmission_power_transformed}
    &- \mathbf{a}_c^H(\varphi_m)\mathbf{P} \mathbf{P}^H \mathbf{a}_c(\varphi_m) =\sum_{k=1}^K \mathbf{p}_k^H \underbrace{\Big(M_c \mathbf{I} - \mathbf{a}_c(\varphi_m)\mathbf{a}_c^H(\varphi_m)\Big)}_{\mathbf{Z}(\varphi_m)}\mathbf{p}_k-M_c\times P_c.
\end{align}
It can be proved that ${\mathbf{Z}(\varphi_m)}$ is a positive semidefinite matrix,
therefore \eqref{eqn:separated_transmission_power_transformed} is a convex function. The problem \eqref{problem:separated_mse_minimization} can be reformulated as
\begin{subequations} \label{problem:separated_mse_minimization_transformed}
    \begin{align}
        \min_{\mathbf{P},\mathbf{R}_{\bf q}} \quad & \rho\sum_{k=1}^K w_k e_k(\mathbf{P},\mathbf{R}_{\bf q})- \mathbf{a}_r^H(\varphi_m)\mathbf{R}_{\bf q}\mathbf{a}_r(\varphi_m) +\sum_{k=1}^K\mathbf{p}_k^H \mathbf{Z}(\varphi_m)\mathbf{p}_k \\
        \mathrm{s.t.} \quad &\mathrm{diag}(\mathbf{R}_{\bf q}) = \frac{P_r \mathbf{1}^{M_r \times 1}}{M_r}, \\ 
        &\mathrm{Tr} (\mathbf{P}\mathbf{P}^H) \leq P_c ,\\ 
        & \mathbf{R}_{\bf q} \succeq 0, {\bf R}_{\bf q} = {\bf R}_{\bf q}^H .
    \end{align}
\end{subequations}

For the fixed $\bf{\Theta}$, the maximization of WSR and probing power can be solved by alternating 
between updating $w_k$ according to \eqref{eqn:seperated_wmmse_weight} and solving \eqref{problem:separated_mse_minimization_transformed}. 
The problem \eqref{problem:separated_mse_minimization_transformed} is an SDP that the CVX toolbox \cite{cvx} can effectively solve.


\subsubsection{Fractional Programming for Passive Beamforming} \label{sec:FP_separated}

For the fixed $\bf{P}$ and ${\bf{R}}_{\bf q}$, the problem \eqref{problem:seperated_problem} is simplified to
\begin{subequations} \label{problem:separated_theta}
    \begin{align}
        \max_{\mathbf{\Theta}} \quad & \sum_{k=1}^K \mu_k \log_2 ( 1 + \gamma_k (\mathbf{\Theta})) 
        \\ \mathrm{s.t.} \quad & {\bf \Theta} = \mathrm{diag}({\bf \Theta}_1,{\bf \Theta}_2,...,{\bf \Theta}_G) ,
        \\ &{\bf \Theta}_g = {\bf \Theta}_g^T, {\bf \Theta}_g^H{\bf \Theta}_g = {\bf I}, \forall g .
    \end{align}
\end{subequations}
The logarithmic fractional form of WSR and the quadratic equality constraint lead to the non-convexity of problem \eqref{problem:separated_theta}.
Nevertheless, we show that it can be tackled based on Lagrangian dual transform \cite{shen2018fractional2}, quadratic transform \cite{shen2018fractional},
and scattering-reactance relationship \cite{shen2020modeling}.

Applying the Lagrangian dual transform proposed in \cite{shen2018fractional2},
\eqref{problem:separated_theta} is equivalent to 
\begin{subequations} \label{problem:separated_theta_dual_transformed}
    \begin{align}
        \max_{\mathbf{\Theta},\boldsymbol{\alpha}} \quad & f(\mathbf{\Theta},\boldsymbol{\alpha}) = \sum_{k=1}^K\mu_k\log_2(1+\alpha_k)-\sum_{k=1}^K\mu_k\alpha_k +\sum_{k=1}^K \frac{\mu_k(1+\alpha_k)\gamma_k}{1+\gamma_k}\\
        \mathrm{s.t.} \quad &  {\bf \Theta} = \mathrm{diag}({\bf \Theta}_1,{\bf \Theta}_2,...,{\bf \Theta}_G) ,
        \\ &{\bf \Theta}_g = {\bf \Theta}_g^T, {\bf \Theta}_g^H{\bf \Theta}_g = {\bf I}, \forall g.
    \end{align}
\end{subequations}

% \begin{align}
%     f(\mathbf{\Theta},\boldsymbol{\alpha})=&\sum_{k=1}^K\mu_k\log_2(1+\alpha_k)-\sum_{k=1}^K\mu_k\alpha_k +\sum_{k=1}^K \frac{\mu_k(1+\alpha_k)\gamma_k}{1+\gamma_k} \label{eq21}
% \end{align}

For fixed $\bf{\Theta}$, \eqref{problem:separated_theta_dual_transformed} is an unconstrained convex optimization
problem with respect to $\boldsymbol{\alpha} = [\alpha_1,...,\alpha_K]$ and the optimal $\boldsymbol{\alpha}^\star$ is given as
\begin{equation} \label{eqn:separated_optimal_alpha}
    \alpha^\star_k=\gamma_k,
\end{equation}
which is obtained from ${\partial f}/{\partial \alpha_k} = 0$.

For the fixed $\boldsymbol{\alpha}$, \eqref{problem:separated_theta_dual_transformed} can be simplified to a fractional form
\begin{subequations} \label{problem:separated_theta_fractional_form}
    \begin{align}
        \max_{\mathbf{\Theta}} \quad & \sum_{k=1}^K \frac{\mu_k(1+\alpha_k)\gamma_k}{1+\gamma_k}\\
        \mathrm{s.t.} \quad & {\bf \Theta} = \mathrm{diag}({\bf \Theta}_1,{\bf \Theta}_2,...,{\bf \Theta}_G) ,
        \\ &{\bf \Theta}_g = {\bf \Theta}_g^T, {\bf \Theta}_g^H{\bf \Theta}_g = {\bf I}, \forall g,
    \end{align}
\end{subequations}
where the logarithm in \eqref{problem:separated_theta} is removed. The problem \eqref{problem:separated_theta_fractional_form} 
is a sum-of-ratios FP problem. To apply the quadratic transform
proposed in \cite{shen2018fractional}, we firstly reformulated the objective function
of \eqref{problem:separated_theta_fractional_form} by letting $\boldsymbol{\theta} = \mathrm{vec}({\bf \Theta}^T)$:

\begin{align}
    g(\boldsymbol{\theta})
    =&\sum_{k=1}^K \frac{\mu_k(1+\alpha_k)\gamma_k}{1+\gamma_k} \nonumber\\ 
    =&\sum_{k=1}^K\frac{\mu_k(1+\alpha_k)|{\bf c}_{k}^H \mathbf{p}_k|^2} {\sum_{j=1}^{K} |{\bf c}_k^H \mathbf{p}_j|^2 + {\bf r}_k^H \mathbf{R}_{\bf q} {\bf r}_k + \sigma^2_n} \nonumber \\ 
    =&\sum_{k=1}^K\frac{\mu_k(1+\alpha_k)|({\boldsymbol \theta}^H {\bf A}_k \mathbf{H}_c + {\bf d}_{c,k}^H) \mathbf{p}_k|^2} {\sum_{j=1}^{K} |({\boldsymbol \theta}^H {\bf A}_k\mathbf{H}_c + {\bf d}_{c,k}^H) \mathbf{p}_j|^2 + ({\boldsymbol \theta}^H {\bf A}_k \mathbf{H}_r + {\bf d}_{r,k}^H) \mathbf{R}_{\bf q} (\mathbf{H}_r^H {\bf A}_k^H {\boldsymbol \theta}+ {\bf d}_{r,k}) + \sigma_n^2} .
\end{align}
The matrix ${\bf A}_k$ is defined as
\begin{equation}
    {\bf A}_k = \left[{\bf J}^0 \tilde{\bf h}_k,  {\bf J}^N \tilde{\bf h}_k,{\bf J}^{2 N} \tilde{\bf h}_k,...,{\bf J}^{(N-1) N} \tilde{\bf h}_k \right],
\end{equation}
where 
\begin{align}
    &\tilde{\bf h}_k = [ {\bf h}_k^H, {\bf 0}_{(N-1)N}^T]^T \in \mathbb{C}^{NN\times 1}, 
    \\ &{\bf J} = \left[ \begin{matrix} {\bf 0}_{NN-1}^T & 0 \\ {\bf I}_{NN-1} & {\bf 0}_{NN-1}\end{matrix} \right].
\end{align}
The $g(\boldsymbol{\theta})$ can be further simplified to
\begin{align}
    g(\boldsymbol{\theta}) = \sum_{k=1}^K \frac{\mu_k(1+\alpha_k) |\boldsymbol{\theta}^H {\bf a}_{k,k} + b_{k,k}|^2}{\sum_{j=1}^K |\boldsymbol{\theta}^H {\bf a}_{j,k} + b_{j,k}|^2 + \boldsymbol{\theta}^H {\bf B}_k \boldsymbol{\theta} +  2 \mathrm{Re} \left\{ \boldsymbol{\theta}^H {\bf f}_k \right\} + m_k},
\end{align}
where
\begin{align} 
    &\mathbf{a}_{j,k}={\bf A}_k \mathbf{H}_c \mathbf{p}_j, 
    \\ &b_{j,k} = {\bf d}_{c,k}^H {\bf p}_j, 
    \\ &\mathbf{B}_k={\bf A}_k\mathbf{H}_r \mathbf{R}_{\bf q} \mathbf{H}_r^H {\bf A}_k^H, 
    \\ &{\bf f}_k = {\bf A}_k {\bf H}_r {\bf R}_{\bf q} {\bf d}_{r,k}, 
    \\ &m_k = {\bf d}_{r,k}^H {\bf R}_{\bf q} {\bf d}_{r.k} + \sigma^2_n.
\end{align}
As $\mathbf{R}_{\bf q}$ is positive semidefinite, it can be written as $\mathbf{R}_{\bf q}=\mathbf{R}_d^H \mathbf{R}_d$.
Therefore, we have $\mathbf{B}_k=(\mathrm{diag}\{\mathbf{h}_k^H\} \mathbf{H}_r \mathbf{R}_d^H)(\mathrm{diag}\{\mathbf{h}_k^H\} \mathbf{H}_r \mathbf{R}_d^H)^H$,
which indicates that $\mathbf{B}_k$ is positive semidefinite.

Applying the quadratic transform of FP proposed in \cite{shen2018fractional}, 
the objective function $g(\boldsymbol{\theta})$ in \eqref{problem:separated_theta_fractional_form} 
is transformed to a new function $h(\boldsymbol{\theta}, {\bf y})$ with ${\bf y} = [y_1,...,y_K]^T$.

\begin{align}
    h(\boldsymbol{\theta}, {\bf y}) = & \sum_{k=1}^K\left( 2 \mathrm{Re} \left\{ y_k^* \sqrt{\mu_k (1+\alpha_k)} \left( \boldsymbol{\theta}^H {\bf a}_{k,k} + b_{k,k} \right) \right\} \right)  \nonumber\\
    & -\sum_{k=1}^K |y_k|^2 \left(\sum_{j=1}^K |\boldsymbol{\theta}^H {\bf a}_{j,k} + b_{j,k}|^2 + \boldsymbol{\theta}^H {\bf B}_k \boldsymbol{\theta} + 2 \mathrm{Re} \left\{ \boldsymbol{\theta}^H {\bf f}_k \right\} + m_k \right). 
\end{align}

The problem \eqref{problem:separated_theta_fractional_form} is equivalent to 
\begin{subequations}\label{problem:separated_theta_quadratic_transform}
    \begin{align} 
        \max_{\boldsymbol{\theta},\mathbf{y}} \quad & h(\boldsymbol{\theta}, {\bf y}) \\  
        \mathrm{s.t.} \quad &  \boldsymbol{\theta} = \mathrm{vec}({\bf \Theta}^T),
        \\ & {\bf \Theta} = \mathrm{diag}({\bf \Theta}_1,{\bf \Theta}_2,...,{\bf \Theta}_G) ,
        \\ &{\bf \Theta}_g = {\bf \Theta}_g^T, {\bf \Theta}_g^H{\bf \Theta}_g = {\bf I}, \forall g,
        \\ & {\bf y} = [y_1,...,y_K]^T, y_k \in \mathbb{C}, \forall k.
    \end{align}
\end{subequations}

For fixed $\boldsymbol{\theta}$, \eqref{problem:separated_theta_quadratic_transform} is an unconstrained convex problem
with respect to $\bf{y}$. By setting $\partial h / \partial y_k$ to zero, we can get the optimal solution:
\begin{equation} \label{eqn:separated_optimal_y}
    y^\star_k = \frac{\sqrt{\mu_k (1+\alpha_k)} \left(\boldsymbol{\theta}^H {\bf a}_{k,k} + b_{k,k} \right)}{\sum_{j=1}^K |\boldsymbol{\theta}^H {\bf a}_{j,k} + b_{j,k}|^2 + \boldsymbol{\theta}^H {\bf B}_k \boldsymbol{\theta} + 2 \mathrm{Re} \left\{ \boldsymbol{\theta}^H {\bf f}_k \right\} + m_k}.
\end{equation}

For fixed $\bf{y}$, the term $|\boldsymbol{\theta}^H {\bf a}_{j,k} + b_{j,k}|^2$ in $h(\boldsymbol{\theta}, {\bf y})$
 can be rephrased as
\begin{align}
    &|\boldsymbol{\theta}^H {\bf a}_{j,k} + b_{j,k}|^2 \nonumber\\
    = &(\boldsymbol{\theta}^H {\bf a}_{j,k} + b_{j,k}) ( {\bf a}_{j,k}^H\boldsymbol{\theta} + b_{j,k}^*) \nonumber \\
    = &\boldsymbol{\theta}^H {\bf a}_{j,k} {\bf a}_{j,k}^H \boldsymbol{\theta} + 2 \mathrm{Re} \left\{ b_{j,k}^* \boldsymbol{\theta}^H {\bf a}_{j,k}  \right\} + |b_{j,k}|^2 .
\end{align}
Substituting this into $h(\boldsymbol{\theta}, {\bf y})$ and dropping the constant term, we get a new objective function
\begin{equation}
    \tilde{h}(\boldsymbol{\theta}) = -\boldsymbol{\theta}^H {\bf U} \boldsymbol{\theta} + 2 \mathrm{Re} \left\{ \boldsymbol{\theta}^H {\bf v} \right\},
\end{equation}
where 
\begin{align}
    &{\bf U} = \sum_{k=1}^K |y_k|^2 \left({\bf B}_k + \sum_{j=1}^K {\bf a}_{j,k} {\bf a}_{j,k}^H  \right), \\
    &{\bf v} = \sum_{k=1}^K \left( y_k^* \sqrt{\mu_k(1+\alpha_k)} {\bf a}_{k,k} - |y_k|^2 \left( {\bf f}_k + \sum_{j=1}^K b_{j,k}^* {\bf a}_{j,k} \right) \right).
\end{align}

Thus, the problem \eqref{problem:separated_theta_quadratic_transform} for fixed ${\bf y}$ is simplified to
\begin{subequations} \label{problem:separated_theta_quadratic_form}
    \begin{align}
        \min_{\boldsymbol{\theta}} \quad & \boldsymbol{\theta}^H {\bf U} \boldsymbol{\theta} - 2 \mathrm{Re} \left\{ \boldsymbol{\theta}^H {\bf v} \right\}\\  
        \mathrm{s.t.} \quad &  \boldsymbol{\theta} = \mathrm{vec}({\bf \Theta}^T)
        \\ &{\bf \Theta} = \mathrm{diag}({\bf \Theta}_1,{\bf \Theta}_2,...,{\bf \Theta}_G), 
        \\ &{\bf \Theta}_g = {\bf \Theta}_g^T, {\bf \Theta}_g^H{\bf \Theta}_g = {\bf I}, \forall g.
    \end{align}
\end{subequations}

As ${\bf{B}}_k$ is positive semidefinite, ${\bf U}$ is also positive semidefinite. Therefore, the objective functions of
\eqref{problem:separated_theta_quadratic_form} is convex. However, constrains are still non-convex. In order to tackle this problem, we follow the 
method proposed in \cite{shen2020modeling}, where the relationship between ${\bf \Theta}_g$ and 
reactance matrix ${\bf Z}_{I,g}$ is exploited, that is
\begin{align}
    &{\bf \Theta}_g = (j{\bf Z}_{I,g} + R_0 {\bf I})^{-1}(j{\bf Z}_{I,g} - R_0 {\bf I}), \forall g, 
    \\ &{\bf Z}_{I,g} = {\bf Z}_{I,g}^T, \forall g,
\end{align}
where $R_0$ refers to the characteristic impedance and is usually set as $R_0=50 \Omega$. The problem \eqref{problem:separated_theta_quadratic_form} can be 
rewritten as
\begin{subequations} \label{problem:separated_theta_unconstraint}
    \begin{align}
        \min_{\boldsymbol{\theta}} \quad & \boldsymbol{\theta}^H {\bf U} \boldsymbol{\theta} - 2 \mathrm{Re} \left\{ \boldsymbol{\theta}^H {\bf v} \right\}\\  
        \mathrm{s.t.} \quad &  \boldsymbol{\theta} = \mathrm{vec}({\bf \Theta}^T)
        \\ &{\bf \Theta} = \mathrm{diag}({\bf \Theta}_1,{\bf \Theta}_2,...,{\bf \Theta}_G),
        \\ &{\bf \Theta}_g = (j{\bf Z}_{I,g} + R_0 {\bf I})^{-1}(j{\bf Z}_{I,g} - R_0 {\bf I}), \forall g, 
        \\ &{\bf Z}_{I,g}={\bf Z}_{I,g}^T, \forall g.
    \end{align}
\end{subequations}
The objective of \eqref{problem:separated_theta_unconstraint} is essentially a function of ${\bf Z}_{I,g}$. Because ${\bf Z}_{I,g}$ can be arbitrary real symmetric
matrix, \eqref{problem:separated_theta_unconstraint} is an unconstrained optimization problem that can be addressed by utilising the Quasi-Newton method 
to optimize the upper triangular part of ${\bf Z}_{I,g}$.

In the special case where $G = N$, i.e., single connected network, the passive beamforming matrix ${\bf \Theta}$ is a diagonal matrix
given as 
\begin{equation}
    {\bf \Theta} = \mathrm{diag} \{ \theta_1,...,\theta_N \},
\end{equation}
where $|\theta_i| \leq 1, i=1,...N$. Following the same path of deriving \eqref{problem:separated_theta_unconstraint}, the following Quadratic Programme (QP)
can be obtained.
\begin{subequations} \label{problem:separated_theta_single}
    \begin{align}
        \min_{\boldsymbol{\breve \theta}} \quad & \boldsymbol{\breve \theta}^H \breve{\bf U} \boldsymbol{\breve \theta} - 2 \mathrm{Re} \left\{ \boldsymbol{\breve \theta}^H \breve{\bf v} \right\}\\  
        \mathrm{s.t.} \quad &  \boldsymbol{\breve \theta} = [\theta_1,...,\theta_N]^T,
        \\ &|\theta_i| \leq 1, \forall i,
    \end{align}
\end{subequations}
where 
\begin{align}
    &\breve{\bf U} = \sum_{k=1}^K |\breve{y}_k|^2 \left(\breve{\bf B}_k + \sum_{j=1}^K \breve{\bf a}_{j,k} \breve{\bf a}_{j,k}^H  \right), \\
    &\breve{\bf v} = \sum_{k=1}^K \left( \breve{y}_k^* \sqrt{\mu_k(1+\alpha_k)} \breve{\bf a}_{k,k} - |\breve{y}_k|^2 \left( \breve{\bf f}_k + \sum_{j=1}^K b_{j,k}^* \breve{\bf a}_{j,k} \right) \right), \\
    \label{eqn:separated_optimal_y_single}
    &\breve{y}_k = \frac{\sqrt{\mu_k (1+\alpha_k)} \left(\boldsymbol{\theta}^H \breve{\bf a}_{k,k} + b_{k,k} \right)}{\sum_{j=1}^K |\boldsymbol{\theta}^H \breve{\bf a}_{j,k} + b_{j,k}|^2 + \boldsymbol{\theta}^H \breve{\bf B}_k \boldsymbol{\theta} + 2 \mathrm{Re} \left\{ \boldsymbol{\theta}^H \breve{\bf f}_k \right\} + m_k},\\
    &\breve{\bf{a}}_{j,k}=\mathrm{diag}\{\mathbf{h}_k^H\} \mathbf{H}_c \mathbf{p}_j, \\
    &\breve{\bf{B}}_k=\mathrm{diag}\{\mathbf{h}_k^H\}\mathbf{H}_r \mathbf{R}_{\bf q} \mathbf{H}_r^H\mathrm{diag}\{\mathbf{h}_k\}, \\
    &\breve{\bf f}_k = \mathrm{diag} \{{\bf h}_k^H\} {\bf H}_r {\bf R}_{\bf q} {\bf d}_{r,k}.\\
\end{align}
The matrix $\breve{\bf U}$ can also be proved to be positive semidefinite, therefore \eqref{problem:separated_theta_single} is a convex QP.
Similar to SDP, QP can also be solved by CVX toolbox \cite{cvx}. 

Therefore, we propose a WMMSE-FP-based AO algorithm to solve the problem \eqref{problem:seperated_problem}.
The details are shown in Algorithm \ref{alg:A}


\begin{algorithm}[htb]
    \caption{WMMSE-FP-based AO Algorithm for Separated Deployment}
    \label{alg:A}
    \begin{algorithmic}[1]
        \REQUIRE{$t = 0$, ${\bf{P}}^{[t]}$, ${\bf{R}}_{\bf q}^{[t]}$, ${\bf{\Theta}}^{[t]}$;}
        \STATE{Calculate $\text{WSR}^{[t]}$ from ${\bf{P}}^{[t]}$, ${\bf{R}}_{\bf q}^{[t]}$ and ${\bf{\Theta}}^{[t]}$;} 
        \REPEAT 
        \STATE{$t = t+1$}
        \STATE{Calculate $g_k^{[t]}$ by \eqref{eqn:separated_mmse_receiver};}
        \STATE{Calculate $w_k^{[t]}$ by \eqref{eqn:seperated_wmmse_weight};}
        \STATE{Update active beamforming ${\bf{P}}^{[t]}$ and radar covariance matrix ${\bf{R}}_{\bf q}^{[t]}$ by solving \eqref{problem:separated_mse_minimization_transformed};}

        \STATE{Calculate $\alpha_k^{[t]}$ by \eqref{eqn:separated_optimal_alpha};}
        \STATE{Calculate $y_k^{[t]}$ by \eqref{eqn:separated_optimal_y} for group/fully connected RIS or $\breve{y}_k^{[t]}$ by \eqref{eqn:separated_optimal_y_single} for single connected RIS;}
        \STATE{Update ${\bf{\Theta}}^{[t]}$ by solving \eqref{problem:separated_theta_unconstraint} for group/fully connected RIS or \eqref{problem:separated_theta_single} for single connected RIS;}

        \STATE{Calculate $\text{WSR}^{[t]}$ from ${\bf{P}}^{[t]}$, ${\bf{R}}_{\bf q}^{[t]}$ and ${\bf{\Theta}}^{[t]}$;}

        \UNTIL{$\left|\text{WSR}^{[t-1]} - \text{WSR}^{[t]}\right| \leq \epsilon$.}

    \end{algorithmic}
\end{algorithm}

\subsection{Algorithm for Shared Deployment}

For fixed ${\bf \Theta}$, we firstly follow the same path in Section \ref{sec:WMMSE_separated} and reformulate \eqref{problem:shared_problem} using WMMSE as
\begin{subequations} \label{problem:shared_mse_minimization}
    \begin{align} 
        \min_{\check{\bf p}_1,...,\check{\bf p}_K} \quad &\rho \sum_{k=1}^K \check{w}_k \check{e}_k(\check{\bf p}_1,...,\check{\bf p}_K) +  \sum_{k=1}^K \check{\bf p}_k^H {\bf Z}(\varphi_m) \check{\bf p}_k  
        \\ \mathrm{s.t.} \quad &\mathrm{diag}\left( \sum_{j=1}^K \check{\bf p}_j \check{\bf p}_j^H \right) = \frac{P}{M} {\bf 1}^{M \times 1}, 
    \end{align}
\end{subequations}
where
\begin{align} 
    &\check{e}_k = |\check{g}_k|^2 \left( \sum_{j=1}^K |\check{\bf c}_k^H \check{\bf p}_j|^2 + \sigma_n^2 \right) - 2\mathrm{Re}\{ g_k \check{\bf c}_k^H \check{\bf p}_k\} + 1, \\
    \label{eqn:shared_mmse_receiver}
    &\check{g}_k^{\mathrm {MMSE}} = \frac{\check{\bf p}_k^H \check{\bf c}_k}{\sum_{j=1}^K |\check{\bf c}_k^H \check{\bf p}_j|^2 + \sigma_n^2},\\
    &\check{e}_k^{\mathrm{MMSE}} = 1-\frac{|\check{\bf c}_k^H \check{\bf p}_k|^2}{\sum_{j=1}^K |\check{\bf c}_k^H \check{\bf p}_j|^2 + \sigma_n^2},\\
    \label{eqn:shared_wmmse_weight}
    &\check{w}_k = \mu_k (\check{e}^{\mathrm{MMSE}})^{-1}, \\
    & {\bf Z}(\varphi_m) = \Big( M{\bf I} - {\bf a}(\varphi_m){\bf a}^H(\varphi_m) \Big).
\end{align}
The non-convexity of this problem is from the quadratic equality constraint. In \cite{xu2020tradeoff}, the authors proposed an MM algorithm to address it.
But the MM is an iterative algorithm, which leads to high complexity when it is nested in an outer iterative loop.
Thus, we propose a new low complexity algorithm based on SDR, which is capable of solving this problem in one step.   
Observing that this problem is an inhomogeneous QCQP, we firstly homogenize this problem.

The $\check{e}_k$ can be rewritten as
\begin{align}
    \check{e}_k &= |\check{g}_k|^2 \left( \sum_{j=1}^K |\check{\bf c}_k^H \check{\bf p}_j|^2 + \sigma_n^2 \right) - 2\mathrm{Re}\{ \check{g}_k \check{\bf c}_k^H \check{\bf p}_k\} + 1 \nonumber
    \\ & = |\check{g}_k|^2 \sum_{j=1}^K \check{\bf p}_j^H \check{\bf c}_k \check{\bf c}_k^H \check{\bf p}_j + |\check{g}_k|^2\sigma_n^2 - \check{g}_k \check{\bf c}_k^H \check{\bf p}_k - \check{g}_k^*\check{\bf p}_k^H \check{\bf c}_k + 1
\end{align}
Omit the constants and reformulate it as
\begin{align} 
    \hat{e}_k & = |\check{g}_k|^2 \sum_{j=1}^K \check{\bf p}_j^H \check{\bf c}_k \check{\bf c}_k^H \check{\bf p}_j  - \check{g}_k \check{\bf c}_k^H \check{\bf p}_k - \check{g}_k^* \check{\bf p}_k^H \check{\bf c}_k \nonumber
    \\ &=|\check{g}_k|^2\sum_{j=1,j\neq k}^K \check{\bf p}_j^H \check{\bf c}_k \check{\bf c}_k^H \check{\bf p}_j +|\check{g}_k|^2 \check{\bf p}_k^H \check{\bf c}_k \check{\bf c}_k^H \check{\bf p}_k - \check{g}_k \check{\bf c}_k^H \check{\bf p}_k - \check{g}_k^* \check{\bf p}_k^H \check{\bf c}_k \nonumber
    \\ & = \sum_{j=1,j \neq k}^K \begin{bmatrix} \hat{\bf p}_j^H & t_j^*\end{bmatrix} \begin{bmatrix} |\check{g}_k|^2 \check{\bf c}_k \check{\bf c}_k^H & {\bf 0} \\ {\bf 0}^T & 0 \end{bmatrix}  \begin{bmatrix} \hat{\bf p}_j \\ t_j\end{bmatrix} + \begin{bmatrix}   \hat{\bf p}_k^H & t_k^*\end{bmatrix}\begin{bmatrix} |\check{g}_k|^2 \check{\bf c}_k \check{\bf c}_k^H & -\check{g}_k^*\check{\bf c}_k \\ -\check{g}_k \check{\bf c}_k^H & 0 \end{bmatrix} \begin{bmatrix} \hat{\bf p}_k \\ t_k \end{bmatrix} 
\end{align}
where $|t_j|^2 = 1$ and $\hat{\bf p}_j = t_j \check{\bf p}_j$.

Similarly, we have
\begin{equation}
    \check{\bf p}_k^H {\bf Z}(\varphi_m) \check{\bf p}_k = \begin{bmatrix} \hat{\bf p}_k^H & t_k^* \end{bmatrix} \begin{bmatrix} {\bf Z}(\varphi_m) & {\bf 0} \\ {\bf 0}^T & 0 \end{bmatrix} \begin{bmatrix} \hat{\bf p}_k \\ t_k \end{bmatrix}
\end{equation}

The constraint can also be rewritten as
\begin{align}
    \mathrm{diag}&\left( \sum_{j=1}^K \check{\bf p}_j \check{\bf p}_j^H \right) = \frac{P}{M}{\bf 1}^{M \times 1} \nonumber\\
    &\Longrightarrow\mathrm{diag} \left( \sum_{j=1}^K \begin{bmatrix} \hat{\bf p}_j \\ t_j\end{bmatrix} \begin{bmatrix} \hat{\bf p}_j^H & t_j^* \end{bmatrix}\right)  = \begin{bmatrix} \frac{P}{M} {\bf 1}^{M \times 1} \\ \sum_{j=1}^K |t_j|^2\end{bmatrix} = \begin{bmatrix} \frac{P}{M} {\bf 1}^{M \times 1} \\ K\end{bmatrix}
\end{align}

Therefore, we can obtain a new homogenous QCQP optimization problem
\begin{subequations} \label{problem:shared_homogeneous_qcqp}
    \begin{align} 
        \min_{\tilde{\bf p}_1,...\tilde{\bf p}_K} \quad & \sum_{k=1}^K \left(\hat{w}_k\sum_{j=1,j \neq k}^K \tilde{\bf p}_j^H {\bf C}_{k,1}\tilde{\bf p}_j + \hat{w}_k\tilde{\bf p}_k^H {\bf C}_{k,2} \tilde{\bf p}_k + \tilde{\bf p}_k^H \hat{\bf Z}(\varphi_m) \tilde{\bf p}_k \right) 
        \\ \mathrm{s.t.} \quad & \mathrm{diag} \left( \sum_{k=1}^K \tilde{\bf p}_k \tilde{\bf p}_k^H \right) = \begin{bmatrix} \frac{P}{M} {\bf 1}^{M \times 1} \\ K\end{bmatrix}, 
        \\ &|[\tilde{\bf p}_k]_{M+1}|^2 =1, k=1,...,K, 
    \end{align}
\end{subequations}
where 
\begin{align} 
    & \hat{w}_k = \rho \check{w}_k,
    \\ & \tilde{\bf p}_k = \begin{bmatrix} \hat{\bf p}_k \\ t_k\end{bmatrix},
    \\ & {\bf C}_{k,1} = \begin{bmatrix} |\check{g}_k|^2 \check{\bf c}_k \check{\bf c}_k^H & {\bf 0} \\ {\bf 0}^T & 0 \end{bmatrix},
    \\ & {\bf C}_{k,2} = \begin{bmatrix} |\check{g}_k|^2 \check{\bf c}_k \check{\bf c}_k^H & -\check{g}_k^* \check{\bf c}_k \\ -\check{g}_k \check{\bf c}_k^H & 0 \end{bmatrix}, 
    \\ &\hat{\bf Z}(\varphi_m) = \begin{bmatrix} {\bf Z}(\varphi_m) & {\bf 0} \\ {\bf 0}^T & 0 \end{bmatrix},
\end{align}
and $[\tilde{\bf p}_k]_{M+1}$ denotes the $(M+1)$-th element of $\tilde{\bf p}_k$.

Letting ${\bf T}_k = \tilde{\bf p}_k \tilde{\bf p}_k^H$ and omitting the rank-one constraint, we obtain the SDR of the 
homogenous QCQP problem:
\begin{subequations} \label{problem:shared_sdr}
    \begin{align} 
        \min_{{\bf T}_1,...,{\bf T}_k} \quad & \sum_{k=1}^K \left(\hat{w}_k \sum_{j=1,j\neq k}^K \mathrm{Tr}({\bf C}_{k,1} {\bf T}_j ) + \hat{w}_k \mathrm{Tr}({\bf C}_{k,2} {\bf T}_k) + \mathrm{Tr} (\hat{\bf Z}(\varphi_m) {\bf T}_k) \right) 
        \\ \mathrm{s.t.} \quad & \mathrm{diag}\left( \sum_{k=1}^K {\bf T}_k \right) = \begin{bmatrix} \frac{P}{M} {\bf 1}^{M \times 1} \\ K\end{bmatrix}, 
        \\ &[{\bf T}_k]_{M+1,M+1} = 1, k=1,...,K, 
        \\ & {\bf T}_k \succeq 0, {\bf T}_k = {\bf T}_k^H, k=1,...,K, 
    \end{align}
\end{subequations}
where $[{\bf T}_k]_{M+1,M+1}$ denotes the entry at $(M+1)$-th row and $(M+1)$-th column.

CVX toolbox can easily handle this problem.
After finding the optimal ${\bf T}_k^\star$, we can use the eigenvalue decomposition or Gaussian randomization method \cite{luo2010semidefinite} to approximate
the solution $\tilde{\bf p}_k^\star$ to the QCQP \eqref{problem:shared_homogeneous_qcqp}. The optimal solution $\check{\bf p}_k^\star$ to \eqref{problem:shared_mse_minimization} can be obtained by 
\begin{equation} \label{eqn:shared_optimal_precoder}
    \check{\bf p}_k^\star = \frac{1}{t^\star_k} \hat{\bf p}_k^\star,
\end{equation}
where $t^\star_k = [\tilde{\bf p}^\star_k]_{M+1}$ and $\hat{\bf p}_k^\star = [\tilde{\bf p}^\star_k]_{1:M}$.

For fixed $\check{\bf p}_k, k=1,...K$, the optimization problem of ${\bf \Theta}$ is similar as that in Section \ref{sec:FP_separated} and the
following optimization problem for group connected RIS can be obtained:
\begin{subequations} \label{problem:shared_theta}
    \begin{align}
        \min_{\boldsymbol{\theta}} \quad & \boldsymbol{\theta}^H \check{\bf U} \boldsymbol{\theta} - 2 \mathrm{Re} \left\{ \boldsymbol{\theta}^H \check{\bf v} \right\}\\  
        \mathrm{s.t.} \quad &  \boldsymbol{\theta} = \mathrm{vec}({\bf \Theta}^T),
        \\ &{\bf \Theta} = \mathrm{diag}({\bf \Theta}_1,{\bf \Theta}_2,...,{\bf \Theta}_G), 
        \\ &{\bf \Theta}_g = (j{\bf Z}_{I,g} + R_0 {\bf I})^{-1}(j{\bf Z}_{I,g} - R_0 {\bf I}), \forall g, 
        \\ &{\bf Z}_{I,g}={\bf Z}_{I,g}^T, \forall g,
    \end{align}
\end{subequations}
where 
\begin{align}
    &\check{\bf U} = \sum_{k=1}^K |\check{y}_k|^2  \sum_{j=1}^K \check{\bf a}_{j,k} \check{\bf a}_{j,k}^H, \\
    &\check{\bf v} = \sum_{k=1}^K \left( \check{y}_k^* \sqrt{\mu_k(1+\check{\alpha}_k)} \check{\bf a}_{k,k} - |\check{y}_k|^2 \sum_{j=1}^K \check{b}_{j,k}^* \check{\bf a}_{j,k} \right),\\
    \label{eqn:shared_optimal_y}
    &\check{y}_k = \frac{\sqrt{\mu_k (1+\check{\alpha}_k)} \left(\boldsymbol{\theta}^H \check{\bf a}_{k,k} + \check{b}_{k,k} \right)}{\sum_{j=1}^K |\boldsymbol{\theta}^H \check{\bf a}_{j,k} + \check{b}_{j,k}|^2  + \sigma_n^2}, \\
    &\check{\bf a}_{j,k}={\bf A}_k \mathbf{H} \check{\bf p}_j,\\
    &\check{b}_{j,k} = {\bf d}_k^H \check{\bf p}_j,\\
    \label{eqn:shared_optimal_alpha}
    &\check{\alpha}_k = \check{\gamma}_k.
\end{align}

% \begin{algorithm}[htb]
%     \caption{WMMSE-FP-based AO Algorithm for Shared Deployment}
%     \label{alg:B}
%     \begin{algorithmic}[1]
%         \REQUIRE{$t = 0$, $\check{\bf{p}}_1^{[t]},...,\check{\bf{p}}_K^{[t]}$, ${\bf{\Theta}}^{[t]}$;}
%         \STATE{Calculate $\text{WSR}^{[t]}$ from $\check{\bf{p}}_1^{[t]},...,\check{\bf{p}}_K^{[t]}$ and ${\bf{\Theta}}^{[t]}$;} 
%         \REPEAT 
%         \STATE{$t = t+1$;}
%         \STATE{Calculate $g_k^{[t]}$ by \eqref{eqn:shared_mmse_receiver};}
%         \STATE{Calculate $w_k^{[t]}$ by \eqref{eqn:shared_wmmse_weight};}
%         \STATE{Calculate ${\bf T}^\star_k$ from \eqref{problem:shared_sdr} and approximate $\tilde{\bf p}^\star_k$ using 
%         eigenvalue decomposition or Gaussian randomization;}
%         \STATE{Update $\check{\bf{p}}_1^{[t]},...,\check{\bf{p}}_K^{[t]}$ by \eqref{eqn:shared_optimal_precoder};}

%         \STATE{Calculate $\alpha_k^{[t]}$ by \eqref{eqn:shared_optimal_alpha};}
%         \STATE{Calculate $y_k^{[t]}$ by \eqref{eqn:shared_optimal_y} for group/fully connected RIS or $\breve{y}_k^{[t]}$ by \eqref{eqn:shared_optimal_y_single} for single connected RIS;}
%         \STATE{update ${\bf{\Theta}}^{[t]}$ by solving \eqref{problem:shared_theta} for group/fully connected RIS or \eqref{problem:shared_theta_single} for single connected RIS;}

%         \STATE{Calculate $\text{WSR}^{[t]}$ from $\check{\bf{p}}_1^{[t]},...,\check{\bf{p}}_K^{[t]}$ and ${\bf{\Theta}}^{[t]}$;} 
%         \UNTIL{$\left|\text{WSR}^{[t-1]} - \text{WSR}^{[t]}\right| \leq \epsilon$.}
%     \end{algorithmic}
% \end{algorithm}

The simplified optimization problem for single connected RIS is
\begin{subequations} \label{problem:shared_theta_single}
    \begin{align}
        \min_{\boldsymbol{\breve \theta}} \quad & \boldsymbol{\breve \theta}^H \bar{\bf U} \boldsymbol{\breve \theta} - 2 \mathrm{Re} \left\{ \boldsymbol{\breve \theta}^H \bar{\bf v} \right\}\\  
        \mathrm{s.t.} \quad &  \boldsymbol{\breve \theta} = [\theta_1,...,\theta_N]^T,
        \\ &|\theta_i| \leq 1, \forall i,
    \end{align}
\end{subequations}
where 
\begin{align}
    &\bar{\bf U} = \sum_{k=1}^K |\bar{y}_k|^2 \sum_{j=1}^K \bar{\bf a}_{j,k} \bar{\bf a}_{j,k}^H,  \\
    &\bar{\bf v} = \sum_{k=1}^K \left( \bar{y}_k^* \sqrt{\mu_k(1+\alpha_k)} \bar{\bf a}_{k,k} - |\bar{y}_k|^2  \sum_{j=1}^K \check{b}_{j,k}^* \bar{\bf a}_{j,k} \right), \\
    \label{eqn:shared_optimal_y_single}
    &\bar{y}_k = \frac{\sqrt{\mu_k (1+\check{\alpha}_k)} \left(\boldsymbol{\theta}^H \bar{\bf a}_{k,k} + \check{b}_{k,k} \right)}{\sum_{j=1}^K |\boldsymbol{\theta}^H \bar{\bf a}_{j,k} + \check{b}_{j,k}|^2  + \sigma_n^2}, \\
    &\bar{\bf{a}}_{j,k}=\mathrm{diag}\{\mathbf{h}_k^H\} \mathbf{H} \check{\mathbf{p}}_j. 
\end{align}

Similarly, \eqref{problem:shared_theta} and \eqref{problem:shared_theta_single} can be respectively solved using Quasi-Newton method in MATLAB and CVX toolbox. 

Thus, we propose another WMMSE-FP-based AO algorithm summarized as Algorithm \ref{alg:B} to solve the
problem \eqref{problem:shared_problem}.

\begin{algorithm}[htb]
    \caption{WMMSE-FP-based AO Algorithm for Shared Deployment}
    \label{alg:B}
    \begin{algorithmic}[1]
        \REQUIRE{$t = 0$, $\check{\bf{p}}_1^{[t]},...,\check{\bf{p}}_K^{[t]}$, ${\bf{\Theta}}^{[t]}$;}
        \STATE{Calculate $\text{WSR}^{[t]}$ from $\check{\bf{p}}_1^{[t]},...,\check{\bf{p}}_K^{[t]}$ and ${\bf{\Theta}}^{[t]}$;} 
        \REPEAT 
        \STATE{$t = t+1$;}
        \STATE{Calculate $g_k^{[t]}$ by \eqref{eqn:shared_mmse_receiver};}
        \STATE{Calculate $w_k^{[t]}$ by \eqref{eqn:shared_wmmse_weight};}
        \STATE{Calculate ${\bf T}^\star_k$ from \eqref{problem:shared_sdr} and approximate $\tilde{\bf p}^\star_k$ using 
        eigenvalue decomposition or Gaussian randomization;}
        \STATE{Update $\check{\bf{p}}_1^{[t]},...,\check{\bf{p}}_K^{[t]}$ by \eqref{eqn:shared_optimal_precoder};}

        \STATE{Calculate $\alpha_k^{[t]}$ by \eqref{eqn:shared_optimal_alpha};}
        \STATE{Calculate $y_k^{[t]}$ by \eqref{eqn:shared_optimal_y} for group/fully connected RIS or $\breve{y}_k^{[t]}$ by \eqref{eqn:shared_optimal_y_single} for single connected RIS;}
        \STATE{update ${\bf{\Theta}}^{[t]}$ by solving \eqref{problem:shared_theta} for group/fully connected RIS or \eqref{problem:shared_theta_single} for single connected RIS;}

        \STATE{Calculate $\text{WSR}^{[t]}$ from $\check{\bf{p}}_1^{[t]},...,\check{\bf{p}}_K^{[t]}$ and ${\bf{\Theta}}^{[t]}$;} 
        \UNTIL{$\left|\text{WSR}^{[t-1]} - \text{WSR}^{[t]}\right| \leq \epsilon$.}
    \end{algorithmic}
\end{algorithm}


